\documentclass{article}
\usepackage[margin=1in]{geometry}
\usepackage[english]{babel}
\usepackage{graphicx}
\usepackage{hyperref}

\title{Operating Systems Lab Project Report}
\author{Alessandro Cheli - Università di Pisa\\ A.A. 2019-2020}



\begin{document}
    \maketitle

    \section{Project Overview}
    This is the full project and not the simplified version.  The following
    list contains an overview of the project structure. Header files are
    implicitly omitted from the list where a C source file is specified:
    \begin{itemize}
        \item \textbf{Source Files:} \texttt{manager.c} and \texttt{supermarket.c} files contain the main
        functions for the corresponding processes. 
        \texttt{lqueue.c} contains a generic (pointer to void) implementation
        of a FIFO queue, which relies on \texttt{linked\_list.c}.
        \texttt{conc\_lqueue.c} contains a concurrent wrapper to the linked
        list implementation, relying on mutexes and condition variables defined
        in the standard POSIX threads library.
        For simplicity, concurrent access to the queue is not done through fine-grained
        locking. Also, as suggested by Arpaci-Dusseau in the book
        \textit{Operating Systems: Three Easy Pieces}, a single lock approach
        may result faster because acquiring and releasing locks for each element of
        the list can introduce significant overhead, therefore each queue
        instance includes a single mutual exclusion lock. I considered an hybrid
        approach where a lock is used every $n$ elements, but discarded the idea
        because it was adding an unnecessary layer of complexity to the project.
        \texttt{cashcust.c} contains definitions of the cashier and customer
        data types, worker threads and miscellaneous methods. They are defined in the
        same file because the data structures and threads reference each other,
        \texttt{ini.c} contains a tiny ANSI C library for loading .ini
        config files. It is the only external dependency in the project and
        the source code repository is available at \url{https://github.com/rxi/ini}.
        It is released under the MIT license.

        \item \textbf{Header Only Files:} \texttt{logger.h} contains logging macros. The macros report the
        line number and surrounding function name of the place of invocation in
        the source code, and allow various log levels to be filtered. The default
        level in production is NOTICE, while DEBUG and NEVER are used for debugging.
        \texttt{config.h} contains macro constants for configuration
        variables defaults and the IPC textual protocol message definitions. \texttt{util.h}
        contains miscellaneous utility functions and macros.
        \texttt{globals.h} contains declarations of global flags
        used for quitting on SIGHUP,
        SIGQUIT and SIGINT. Those flags are only set by signal handlers.
        \item \textbf{Config Files:} The \texttt{manager} and
        \texttt{supermarket} executables accept the \texttt{-c} command line option
        that specifies the path to a .ini configuration files. If some of the
        values are not defined in the .ini file, sane defaults are included.
        Example configuration files, also used in tests, can be found in the \texttt{examples/} folder.
        \item \textbf{Shell scripts:} \texttt{memplot.sh} contains a script for memory profiling and plotting
        through \texttt{GNUplot} for this report.
        \texttt{analisi.sh} accepts a supermarket log file and produces
        a short report by using the echo, cat, grep, sed, awk, cut, tr, sort, uniq and bc
        UNIX utilities.
        \texttt{autotexrebuild.sh} is a simple shell script used in
        development that rebuilds and shows the \LaTeX \@ report as soon as it
        is modified by using \texttt{inotifywait}.

    \end{itemize}


    \section{Design Choices}
    The queue data structure is generic and therefore is used for both message
    exchange between threads and for representing customers enqueued to cashiers.
    The concurrent FIFO queue data structure has methods for both blocking
    and non-blocking dequeueing. The nonblocking method reports
    an error when the queue is empty, instead of waiting on a condition variable.
    Signal handling is done synchronously in the manager process by a designated
    thread, which waits on the masked signals using the \texttt{sigwait(3)}
    function. This was needed because the manager process must forward
    signals to connected client processes by using \texttt{kill(2)}, and
    this required access to a mutally excluded array of process IDs. Instead, the signal
    handler in the supermarket is a normal signal handler and is registered
    by using the \texttt{sigaction} system call, as it only needs to set
    two global flags of type \texttt{volatile sig\_atomic\_t}, which are in turn
    used by other parts of the application to check every loop iteration if
    the current thread should be terminated, either by emptying the cashier queues
    or destroying them brutally. 


    In the supermarket process all cashiers and customers are active entities, represented
    by a data structure and by a corresponding worker thread. Cashier threads are terminated
    and joined when they are closed, and threads are created when cashiers are opened.
    Customers threads are created when they enter the supermarket and destroyed when they
    exit. The supermarket process also contains four additional threads. There are
    two threads designated for message handling:
    \texttt{inmsg\_worker} and \texttt{outmsg\_worker}. The former reads messages
    from the socket and applies the manager's decisions of opening or closing cashiers
    and allowing customers out (which are always allowed). To do so, the inbound message
    worker must have access to most data and synchronization structures.
    The outbound message worker
    instead, simply reads messages from a monolithic concurrent queue used in most threads in the process
    and sends the messages to the manager process through the UNIX socket,
    deallocating the consumed message buffers after
    a failed or successful write on the socket.


    The other two helper threads in the supermarket process,
    gather data from queues at regular, configurable intervals.
    The cashier poller thread regularly
    enqueues a message for the manager that contains the size of every cashier's queue (size is -1 if
    the cashier is closed).
    Upon receiving a queue size poll, the manager undergoes a simple decision process
    and answers accordingly to the undercrowded/overcrowded tresholds (called S2 and S1
    in the specification).
    Priority is given to overcrowding (opening new cashiers when needed).
    The customer re-enqueue worker thread, iterates through every cashier's queue
    at regular intervals. Each enqueued customer has a fixed random chance to be
    removed from the queue and be re-enqueued to an open cashier with the shortest
    queue.
    The algorithm for choosing a cashier is in fact very simple. It just chooses
    the cashier which queue is the shortest. To do so, customers must have
    access to the whole array of cashiers data structures.

    Only MT safe library functions have been used in multithreaded environments.
    Random values are obtained with \texttt{rand\_r} and seeds are different for
    every thread. Time differences are measured using \texttt{clock()};
    
    \section{IPC}
    As requested in the full project specification, IPC is achieved between the
    manager and supermarket processes by using an UNIX socket. The manager acts as
    the server and can handle 2 clients at the same time by default, through a
    manager-worker scheme, bounded by a shared variable counting active connections.
    Regarding the manager-supermarket IPC I have opted for a simple textual protocol
    because, although less efficient than direct binary communication, it is
    portable and easily debuggable by using standard command line tools like
    netcat, and could be also used for TCP/IP sockets. There is a short handshake
    process that must go on when a client connects to a manager server.
    This is because the manager needs to keep track of the PIDs of the supermarket
    processes in case it needs to forward a SIGHUP, SIGQUIT or SIGINT signal.
    After receiving a poll, the manager may decide to open or close a cashier, or to
    not do anything.
    With the values used for testing, it seems that the number of open cashiers tends
    to stabilize, as well as the average number of customers enqueued to a cashier
    at any given time. Other than queue size polls and cashier opening and closing
    orders, IPC is also needed when customers have to ask to "get out".
    Before any customer thread can terminate normally, it has to follow a
    "let me out - OK" communication scheme with the manager process.
    Customers are always allowed out of the supermarket.
        
    

    \section{Tools Used and Debugging} I chose \texttt{clang} version 10
    as the development compiler because of the presence
    of additional debugging utilities such as the thread sanitizer, which
    has turned out an helpful tool for debugging data races and reported
    many potential bugs. The project is submitted with \texttt{gcc} as default
    because it is readily available on any system this project might be tested
    on. The default Makefile target builds "production"
    executables without debugging symbols, optimized with the \texttt{-O3}
    flag.  To get code quality reports I have used static code analyzers such
    as \texttt{clang-analyzer} and \texttt{cppcheck}. \texttt{clang-analyzer}
    includes an utility called \texttt{scan-build} which allows to perform
    static analysis on the code at build time, wrapping the \texttt{make}
    command. Executables have been extensively tested for memory leaks by
    using \texttt{valgrind}.  Data races have been reported by both static
    analyzers and clang's thread sanitizer. Deadlocks have been debugged by
    using \texttt{gdb}. \texttt{gdb} was signaled \texttt{SIGCONT} to pause
    the execution of a process when one or more threads got stuck waiting on
    a condition variable or a mutex. Thread backtraces were then inspected
    without the need for classic breakpoint debugging. \texttt{tectonic}
    was used as the \LaTeX\@ compiler for this report.

    \section{Testing} The project has been tested on different Linux
    distributions. It has not been tested on macOS and is not guaranteed
    to be portable because some syscall options have been used where the
    manual specifies that those options were introduced in Linux. Valgrind
    was used to make sure that there were no "definite" leaks. There may be
    indirect leaks when terminating the processes during the execution of
    some standard library functions.
    Bad things could happen if the manager and supermarket communicate
    while configured with different parameters. For simplicity, I assumed that no protection
    mechanism is needed and that the user will test the project only by using
    the same configuration file. Another method could be that the configuration
    values used by both processes are the same when establishing a connection.


    \section{Memory Usage Plots}
    To check for the absence of substantial memory leaks, memory usage was
    tracked and plotted using \texttt{GNUplot}. The script is available in
    \texttt{memtest.sh}. For the purpose of plotting the following two figures,
    both programs were tested by using large quantities of threads and resources.
    Units are in KB.
   
    \begin{figure}[htbp]
        \begin{center}
        % GNUPLOT: LaTeX picture
\setlength{\unitlength}{0.240900pt}
\ifx\plotpoint\undefined\newsavebox{\plotpoint}\fi
\sbox{\plotpoint}{\rule[-0.200pt]{0.400pt}{0.400pt}}%
\begin{picture}(1500,900)(0,0)
\sbox{\plotpoint}{\rule[-0.200pt]{0.400pt}{0.400pt}}%
\put(131,131){\makebox(0,0)[r]{$0$}}
\put(151.0,131.0){\rule[-0.200pt]{4.818pt}{0.400pt}}
\put(131,203){\makebox(0,0)[r]{$100$}}
\put(151.0,203.0){\rule[-0.200pt]{4.818pt}{0.400pt}}
\put(131,274){\makebox(0,0)[r]{$200$}}
\put(151.0,274.0){\rule[-0.200pt]{4.818pt}{0.400pt}}
\put(131,346){\makebox(0,0)[r]{$300$}}
\put(151.0,346.0){\rule[-0.200pt]{4.818pt}{0.400pt}}
\put(131,418){\makebox(0,0)[r]{$400$}}
\put(151.0,418.0){\rule[-0.200pt]{4.818pt}{0.400pt}}
\put(131,489){\makebox(0,0)[r]{$500$}}
\put(151.0,489.0){\rule[-0.200pt]{4.818pt}{0.400pt}}
\put(131,561){\makebox(0,0)[r]{$600$}}
\put(151.0,561.0){\rule[-0.200pt]{4.818pt}{0.400pt}}
\put(131,633){\makebox(0,0)[r]{$700$}}
\put(151.0,633.0){\rule[-0.200pt]{4.818pt}{0.400pt}}
\put(131,704){\makebox(0,0)[r]{$800$}}
\put(151.0,704.0){\rule[-0.200pt]{4.818pt}{0.400pt}}
\put(131,776){\makebox(0,0)[r]{$900$}}
\put(151.0,776.0){\rule[-0.200pt]{4.818pt}{0.400pt}}
\put(151.0,131.0){\rule[-0.200pt]{0.400pt}{4.818pt}}
\put(151,90){\makebox(0,0){$0$}}
\put(151.0,756.0){\rule[-0.200pt]{0.400pt}{4.818pt}}
\put(289.0,131.0){\rule[-0.200pt]{0.400pt}{4.818pt}}
\put(289,90){\makebox(0,0){$500$}}
\put(289.0,756.0){\rule[-0.200pt]{0.400pt}{4.818pt}}
\put(428.0,131.0){\rule[-0.200pt]{0.400pt}{4.818pt}}
\put(428,90){\makebox(0,0){$1000$}}
\put(428.0,756.0){\rule[-0.200pt]{0.400pt}{4.818pt}}
\put(566.0,131.0){\rule[-0.200pt]{0.400pt}{4.818pt}}
\put(566,90){\makebox(0,0){$1500$}}
\put(566.0,756.0){\rule[-0.200pt]{0.400pt}{4.818pt}}
\put(705.0,131.0){\rule[-0.200pt]{0.400pt}{4.818pt}}
\put(705,90){\makebox(0,0){$2000$}}
\put(705.0,756.0){\rule[-0.200pt]{0.400pt}{4.818pt}}
\put(843.0,131.0){\rule[-0.200pt]{0.400pt}{4.818pt}}
\put(843,90){\makebox(0,0){$2500$}}
\put(843.0,756.0){\rule[-0.200pt]{0.400pt}{4.818pt}}
\put(981.0,131.0){\rule[-0.200pt]{0.400pt}{4.818pt}}
\put(981,90){\makebox(0,0){$3000$}}
\put(981.0,756.0){\rule[-0.200pt]{0.400pt}{4.818pt}}
\put(1120.0,131.0){\rule[-0.200pt]{0.400pt}{4.818pt}}
\put(1120,90){\makebox(0,0){$3500$}}
\put(1120.0,756.0){\rule[-0.200pt]{0.400pt}{4.818pt}}
\put(1258.0,131.0){\rule[-0.200pt]{0.400pt}{4.818pt}}
\put(1258,90){\makebox(0,0){$4000$}}
\put(1258.0,756.0){\rule[-0.200pt]{0.400pt}{4.818pt}}
\put(1278,131){\makebox(0,0)[l]{$0$}}
\put(1238.0,131.0){\rule[-0.200pt]{4.818pt}{0.400pt}}
\put(1278,203){\makebox(0,0)[l]{$10000$}}
\put(1238.0,203.0){\rule[-0.200pt]{4.818pt}{0.400pt}}
\put(1278,274){\makebox(0,0)[l]{$20000$}}
\put(1238.0,274.0){\rule[-0.200pt]{4.818pt}{0.400pt}}
\put(1278,346){\makebox(0,0)[l]{$30000$}}
\put(1238.0,346.0){\rule[-0.200pt]{4.818pt}{0.400pt}}
\put(1278,418){\makebox(0,0)[l]{$40000$}}
\put(1238.0,418.0){\rule[-0.200pt]{4.818pt}{0.400pt}}
\put(1278,489){\makebox(0,0)[l]{$50000$}}
\put(1238.0,489.0){\rule[-0.200pt]{4.818pt}{0.400pt}}
\put(1278,561){\makebox(0,0)[l]{$60000$}}
\put(1238.0,561.0){\rule[-0.200pt]{4.818pt}{0.400pt}}
\put(1278,633){\makebox(0,0)[l]{$70000$}}
\put(1238.0,633.0){\rule[-0.200pt]{4.818pt}{0.400pt}}
\put(1278,704){\makebox(0,0)[l]{$80000$}}
\put(1238.0,704.0){\rule[-0.200pt]{4.818pt}{0.400pt}}
\put(1278,776){\makebox(0,0)[l]{$90000$}}
\put(1238.0,776.0){\rule[-0.200pt]{4.818pt}{0.400pt}}
\put(151.0,131.0){\rule[-0.200pt]{0.400pt}{155.380pt}}
\put(151.0,131.0){\rule[-0.200pt]{266.676pt}{0.400pt}}
\put(1258.0,131.0){\rule[-0.200pt]{0.400pt}{155.380pt}}
\put(151.0,776.0){\rule[-0.200pt]{266.676pt}{0.400pt}}
\put(36,453){\makebox(0,0){RSS}}
\put(1418,453){\makebox(0,0){VSZ}}
\put(704,29){\makebox(0,0){Samples of 0.2 seconds}}
\put(704,838){\makebox(0,0){Memory usage graph of process ./manager, PID 3976}}
\put(1098,212){\makebox(0,0)[r]{RSS}}
\put(1118.0,212.0){\rule[-0.200pt]{24.090pt}{0.400pt}}
\put(151,707){\usebox{\plotpoint}}
\put(151,707){\usebox{\plotpoint}}
\put(151.0,707.0){\rule[-0.200pt]{251.500pt}{0.400pt}}
\sbox{\plotpoint}{\rule[-0.500pt]{1.000pt}{1.000pt}}%
\sbox{\plotpoint}{\rule[-0.200pt]{0.400pt}{0.400pt}}%
\put(1098,171){\makebox(0,0)[r]{VSZ}}
\sbox{\plotpoint}{\rule[-0.500pt]{1.000pt}{1.000pt}}%
\multiput(1118,171)(20.756,0.000){5}{\usebox{\plotpoint}}
\put(1218,171){\usebox{\plotpoint}}
\put(151,207){\usebox{\plotpoint}}
\put(151.00,207.00){\usebox{\plotpoint}}
\put(171.76,207.00){\usebox{\plotpoint}}
\put(192.51,207.00){\usebox{\plotpoint}}
\put(213.27,207.00){\usebox{\plotpoint}}
\put(234.02,207.00){\usebox{\plotpoint}}
\put(254.78,207.00){\usebox{\plotpoint}}
\put(275.53,207.00){\usebox{\plotpoint}}
\put(296.29,207.00){\usebox{\plotpoint}}
\put(317.04,207.00){\usebox{\plotpoint}}
\put(337.80,207.00){\usebox{\plotpoint}}
\put(358.56,207.00){\usebox{\plotpoint}}
\multiput(366,207)(0.039,20.755){25}{\usebox{\plotpoint}}
\put(370.32,736.00){\usebox{\plotpoint}}
\put(391.08,736.00){\usebox{\plotpoint}}
\put(411.83,736.00){\usebox{\plotpoint}}
\put(432.59,736.00){\usebox{\plotpoint}}
\put(453.34,736.00){\usebox{\plotpoint}}
\put(474.10,736.00){\usebox{\plotpoint}}
\put(494.85,736.00){\usebox{\plotpoint}}
\put(515.61,736.00){\usebox{\plotpoint}}
\put(536.36,736.00){\usebox{\plotpoint}}
\put(557.12,736.00){\usebox{\plotpoint}}
\put(577.88,736.00){\usebox{\plotpoint}}
\put(598.63,736.00){\usebox{\plotpoint}}
\put(619.39,736.00){\usebox{\plotpoint}}
\put(640.14,736.00){\usebox{\plotpoint}}
\put(660.90,736.00){\usebox{\plotpoint}}
\put(681.65,736.00){\usebox{\plotpoint}}
\put(702.41,736.00){\usebox{\plotpoint}}
\put(723.16,736.00){\usebox{\plotpoint}}
\put(743.92,736.00){\usebox{\plotpoint}}
\put(764.67,736.00){\usebox{\plotpoint}}
\put(785.43,736.00){\usebox{\plotpoint}}
\put(806.19,736.00){\usebox{\plotpoint}}
\put(826.94,736.00){\usebox{\plotpoint}}
\put(847.70,736.00){\usebox{\plotpoint}}
\put(868.45,736.00){\usebox{\plotpoint}}
\put(889.21,736.00){\usebox{\plotpoint}}
\put(909.96,736.00){\usebox{\plotpoint}}
\put(930.72,736.00){\usebox{\plotpoint}}
\put(951.47,736.00){\usebox{\plotpoint}}
\put(972.23,736.00){\usebox{\plotpoint}}
\put(992.98,736.00){\usebox{\plotpoint}}
\put(1013.74,736.00){\usebox{\plotpoint}}
\put(1034.50,736.00){\usebox{\plotpoint}}
\put(1055.25,736.00){\usebox{\plotpoint}}
\put(1076.01,736.00){\usebox{\plotpoint}}
\put(1096.76,736.00){\usebox{\plotpoint}}
\put(1117.52,736.00){\usebox{\plotpoint}}
\put(1138.27,736.00){\usebox{\plotpoint}}
\put(1159.03,736.00){\usebox{\plotpoint}}
\put(1179.78,736.00){\usebox{\plotpoint}}
\put(1195,736){\usebox{\plotpoint}}
\sbox{\plotpoint}{\rule[-0.200pt]{0.400pt}{0.400pt}}%
\put(151.0,131.0){\rule[-0.200pt]{0.400pt}{155.380pt}}
\put(151.0,131.0){\rule[-0.200pt]{266.676pt}{0.400pt}}
\put(1258.0,131.0){\rule[-0.200pt]{0.400pt}{155.380pt}}
\put(151.0,776.0){\rule[-0.200pt]{266.676pt}{0.400pt}}
\end{picture}

        \end{center}
    \end{figure}

    \begin{figure}[htbp]
        \begin{center}
        % GNUPLOT: LaTeX picture
\setlength{\unitlength}{0.240900pt}
\ifx\plotpoint\undefined\newsavebox{\plotpoint}\fi
\sbox{\plotpoint}{\rule[-0.200pt]{0.400pt}{0.400pt}}%
\begin{picture}(1500,900)(0,0)
\sbox{\plotpoint}{\rule[-0.200pt]{0.400pt}{0.400pt}}%
\put(151,131){\makebox(0,0)[r]{$0$}}
\put(171.0,131.0){\rule[-0.200pt]{4.818pt}{0.400pt}}
\put(151,223){\makebox(0,0)[r]{$500$}}
\put(171.0,223.0){\rule[-0.200pt]{4.818pt}{0.400pt}}
\put(151,315){\makebox(0,0)[r]{$1000$}}
\put(171.0,315.0){\rule[-0.200pt]{4.818pt}{0.400pt}}
\put(151,407){\makebox(0,0)[r]{$1500$}}
\put(171.0,407.0){\rule[-0.200pt]{4.818pt}{0.400pt}}
\put(151,500){\makebox(0,0)[r]{$2000$}}
\put(171.0,500.0){\rule[-0.200pt]{4.818pt}{0.400pt}}
\put(151,592){\makebox(0,0)[r]{$2500$}}
\put(171.0,592.0){\rule[-0.200pt]{4.818pt}{0.400pt}}
\put(151,684){\makebox(0,0)[r]{$3000$}}
\put(171.0,684.0){\rule[-0.200pt]{4.818pt}{0.400pt}}
\put(151,776){\makebox(0,0)[r]{$3500$}}
\put(171.0,776.0){\rule[-0.200pt]{4.818pt}{0.400pt}}
\put(171.0,131.0){\rule[-0.200pt]{0.400pt}{4.818pt}}
\put(171,90){\makebox(0,0){$0$}}
\put(171.0,756.0){\rule[-0.200pt]{0.400pt}{4.818pt}}
\put(321.0,131.0){\rule[-0.200pt]{0.400pt}{4.818pt}}
\put(321,90){\makebox(0,0){$1000$}}
\put(321.0,756.0){\rule[-0.200pt]{0.400pt}{4.818pt}}
\put(470.0,131.0){\rule[-0.200pt]{0.400pt}{4.818pt}}
\put(470,90){\makebox(0,0){$2000$}}
\put(470.0,756.0){\rule[-0.200pt]{0.400pt}{4.818pt}}
\put(620.0,131.0){\rule[-0.200pt]{0.400pt}{4.818pt}}
\put(620,90){\makebox(0,0){$3000$}}
\put(620.0,756.0){\rule[-0.200pt]{0.400pt}{4.818pt}}
\put(769.0,131.0){\rule[-0.200pt]{0.400pt}{4.818pt}}
\put(769,90){\makebox(0,0){$4000$}}
\put(769.0,756.0){\rule[-0.200pt]{0.400pt}{4.818pt}}
\put(919.0,131.0){\rule[-0.200pt]{0.400pt}{4.818pt}}
\put(919,90){\makebox(0,0){$5000$}}
\put(919.0,756.0){\rule[-0.200pt]{0.400pt}{4.818pt}}
\put(1068.0,131.0){\rule[-0.200pt]{0.400pt}{4.818pt}}
\put(1068,90){\makebox(0,0){$6000$}}
\put(1068.0,756.0){\rule[-0.200pt]{0.400pt}{4.818pt}}
\put(1218.0,131.0){\rule[-0.200pt]{0.400pt}{4.818pt}}
\put(1218,90){\makebox(0,0){$7000$}}
\put(1218.0,756.0){\rule[-0.200pt]{0.400pt}{4.818pt}}
\put(1238,131){\makebox(0,0)[l]{$0$}}
\put(1198.0,131.0){\rule[-0.200pt]{4.818pt}{0.400pt}}
\put(1238,203){\makebox(0,0)[l]{$500000$}}
\put(1198.0,203.0){\rule[-0.200pt]{4.818pt}{0.400pt}}
\put(1238,274){\makebox(0,0)[l]{$1\times10^{6}$}}
\put(1198.0,274.0){\rule[-0.200pt]{4.818pt}{0.400pt}}
\put(1238,346){\makebox(0,0)[l]{$1.5\times10^{6}$}}
\put(1198.0,346.0){\rule[-0.200pt]{4.818pt}{0.400pt}}
\put(1238,418){\makebox(0,0)[l]{$2\times10^{6}$}}
\put(1198.0,418.0){\rule[-0.200pt]{4.818pt}{0.400pt}}
\put(1238,489){\makebox(0,0)[l]{$2.5\times10^{6}$}}
\put(1198.0,489.0){\rule[-0.200pt]{4.818pt}{0.400pt}}
\put(1238,561){\makebox(0,0)[l]{$3\times10^{6}$}}
\put(1198.0,561.0){\rule[-0.200pt]{4.818pt}{0.400pt}}
\put(1238,633){\makebox(0,0)[l]{$3.5\times10^{6}$}}
\put(1198.0,633.0){\rule[-0.200pt]{4.818pt}{0.400pt}}
\put(1238,704){\makebox(0,0)[l]{$4\times10^{6}$}}
\put(1198.0,704.0){\rule[-0.200pt]{4.818pt}{0.400pt}}
\put(1238,776){\makebox(0,0)[l]{$4.5\times10^{6}$}}
\put(1198.0,776.0){\rule[-0.200pt]{4.818pt}{0.400pt}}
\put(171.0,131.0){\rule[-0.200pt]{0.400pt}{155.380pt}}
\put(171.0,131.0){\rule[-0.200pt]{252.222pt}{0.400pt}}
\put(1218.0,131.0){\rule[-0.200pt]{0.400pt}{155.380pt}}
\put(171.0,776.0){\rule[-0.200pt]{252.222pt}{0.400pt}}
\put(36,453){\makebox(0,0){RSS}}
\put(1418,453){\makebox(0,0){VSZ}}
\put(694,29){\makebox(0,0){Samples of 0.2 seconds}}
\put(694,838){\makebox(0,0){Memory usage graph of process ./supermarket, PID 6097}}
\put(1058,212){\makebox(0,0)[r]{RSS}}
\put(1078.0,212.0){\rule[-0.200pt]{24.090pt}{0.400pt}}
\put(171,531){\usebox{\plotpoint}}
\put(171,531){\usebox{\plotpoint}}
\put(171,531){\usebox{\plotpoint}}
\put(171,531){\usebox{\plotpoint}}
\put(171.0,531.0){\usebox{\plotpoint}}
\put(172.0,531.0){\rule[-0.200pt]{0.400pt}{3.373pt}}
\put(172.0,545.0){\rule[-0.200pt]{0.482pt}{0.400pt}}
\put(174.0,545.0){\rule[-0.200pt]{0.400pt}{1.204pt}}
\put(174.0,550.0){\rule[-0.200pt]{0.723pt}{0.400pt}}
\put(177.0,550.0){\rule[-0.200pt]{0.400pt}{0.964pt}}
\put(177.0,554.0){\usebox{\plotpoint}}
\put(178.0,554.0){\rule[-0.200pt]{0.400pt}{1.686pt}}
\put(178.0,561.0){\usebox{\plotpoint}}
\put(179.0,561.0){\rule[-0.200pt]{0.400pt}{0.482pt}}
\put(179.0,563.0){\usebox{\plotpoint}}
\put(179.67,567){\rule{0.400pt}{0.723pt}}
\multiput(179.17,567.00)(1.000,1.500){2}{\rule{0.400pt}{0.361pt}}
\put(180.0,563.0){\rule[-0.200pt]{0.400pt}{0.964pt}}
\put(181,570){\usebox{\plotpoint}}
\put(181,570){\usebox{\plotpoint}}
\put(181.0,570.0){\rule[-0.200pt]{0.400pt}{1.927pt}}
\put(181.0,578.0){\usebox{\plotpoint}}
\put(181.67,585){\rule{0.400pt}{0.482pt}}
\multiput(181.17,585.00)(1.000,1.000){2}{\rule{0.400pt}{0.241pt}}
\put(182.0,578.0){\rule[-0.200pt]{0.400pt}{1.686pt}}
\put(183,587){\usebox{\plotpoint}}
\put(183,587){\usebox{\plotpoint}}
\put(183,587){\usebox{\plotpoint}}
\put(182.67,590){\rule{0.400pt}{0.482pt}}
\multiput(182.17,590.00)(1.000,1.000){2}{\rule{0.400pt}{0.241pt}}
\put(183.0,587.0){\rule[-0.200pt]{0.400pt}{0.723pt}}
\put(184,592){\usebox{\plotpoint}}
\put(184,592){\usebox{\plotpoint}}
\put(184.0,592.0){\rule[-0.200pt]{0.400pt}{0.964pt}}
\put(184.0,596.0){\usebox{\plotpoint}}
\put(184.67,598){\rule{0.400pt}{1.204pt}}
\multiput(184.17,598.00)(1.000,2.500){2}{\rule{0.400pt}{0.602pt}}
\put(185.0,596.0){\rule[-0.200pt]{0.400pt}{0.482pt}}
\put(186.0,603.0){\rule[-0.200pt]{0.400pt}{0.482pt}}
\put(186.0,605.0){\usebox{\plotpoint}}
\put(187.0,605.0){\rule[-0.200pt]{0.400pt}{0.482pt}}
\put(187.0,607.0){\usebox{\plotpoint}}
\put(187.67,609){\rule{0.400pt}{0.723pt}}
\multiput(187.17,609.00)(1.000,1.500){2}{\rule{0.400pt}{0.361pt}}
\put(188.0,607.0){\rule[-0.200pt]{0.400pt}{0.482pt}}
\put(189.0,612.0){\rule[-0.200pt]{0.400pt}{0.482pt}}
\put(189.0,614.0){\rule[-0.200pt]{0.482pt}{0.400pt}}
\put(190.67,616){\rule{0.400pt}{0.482pt}}
\multiput(190.17,616.00)(1.000,1.000){2}{\rule{0.400pt}{0.241pt}}
\put(191.0,614.0){\rule[-0.200pt]{0.400pt}{0.482pt}}
\put(192,618){\usebox{\plotpoint}}
\put(192.0,618.0){\rule[-0.200pt]{0.400pt}{1.204pt}}
\put(192.0,623.0){\usebox{\plotpoint}}
\put(193.0,623.0){\rule[-0.200pt]{0.400pt}{0.482pt}}
\put(193.0,625.0){\usebox{\plotpoint}}
\put(194.0,625.0){\rule[-0.200pt]{0.400pt}{0.964pt}}
\put(194.0,629.0){\usebox{\plotpoint}}
\put(195.0,629.0){\rule[-0.200pt]{0.400pt}{1.686pt}}
\put(195.0,636.0){\rule[-0.200pt]{0.482pt}{0.400pt}}
\put(197.0,636.0){\rule[-0.200pt]{0.400pt}{0.482pt}}
\put(197.0,638.0){\rule[-0.200pt]{0.723pt}{0.400pt}}
\put(200.0,638.0){\rule[-0.200pt]{0.400pt}{0.723pt}}
\put(200.0,641.0){\usebox{\plotpoint}}
\put(201.0,641.0){\rule[-0.200pt]{0.400pt}{0.482pt}}
\put(201.0,643.0){\rule[-0.200pt]{0.482pt}{0.400pt}}
\put(203.0,643.0){\rule[-0.200pt]{0.400pt}{0.723pt}}
\put(203.0,646.0){\usebox{\plotpoint}}
\put(204.0,646.0){\rule[-0.200pt]{0.400pt}{0.964pt}}
\put(205.67,650){\rule{0.400pt}{0.482pt}}
\multiput(205.17,650.00)(1.000,1.000){2}{\rule{0.400pt}{0.241pt}}
\put(204.0,650.0){\rule[-0.200pt]{0.482pt}{0.400pt}}
\put(207,652){\usebox{\plotpoint}}
\put(207,652){\usebox{\plotpoint}}
\put(207,652){\usebox{\plotpoint}}
\put(207,652){\usebox{\plotpoint}}
\put(207,652){\usebox{\plotpoint}}
\put(207,652){\usebox{\plotpoint}}
\put(207.0,652.0){\rule[-0.200pt]{0.723pt}{0.400pt}}
\put(210.0,652.0){\usebox{\plotpoint}}
\put(210.0,653.0){\rule[-0.200pt]{0.482pt}{0.400pt}}
\put(212.0,653.0){\usebox{\plotpoint}}
\put(220,653.67){\rule{0.241pt}{0.400pt}}
\multiput(220.00,653.17)(0.500,1.000){2}{\rule{0.120pt}{0.400pt}}
\put(212.0,654.0){\rule[-0.200pt]{1.927pt}{0.400pt}}
\put(221,655){\usebox{\plotpoint}}
\put(221,655){\usebox{\plotpoint}}
\put(221,655){\usebox{\plotpoint}}
\put(221,655){\usebox{\plotpoint}}
\put(221,655){\usebox{\plotpoint}}
\put(221,655){\usebox{\plotpoint}}
\put(221.0,655.0){\rule[-0.200pt]{0.482pt}{0.400pt}}
\put(223.0,655.0){\usebox{\plotpoint}}
\put(223.0,656.0){\rule[-0.200pt]{0.482pt}{0.400pt}}
\put(225.0,656.0){\usebox{\plotpoint}}
\put(225.0,657.0){\rule[-0.200pt]{0.482pt}{0.400pt}}
\put(227.0,657.0){\rule[-0.200pt]{0.400pt}{0.723pt}}
\put(227.0,660.0){\rule[-0.200pt]{1.445pt}{0.400pt}}
\put(233.0,660.0){\usebox{\plotpoint}}
\put(233.0,661.0){\rule[-0.200pt]{1.445pt}{0.400pt}}
\put(239.0,661.0){\usebox{\plotpoint}}
\put(239.0,662.0){\rule[-0.200pt]{0.482pt}{0.400pt}}
\put(241.0,662.0){\rule[-0.200pt]{0.400pt}{0.482pt}}
\put(241.0,664.0){\rule[-0.200pt]{0.482pt}{0.400pt}}
\put(243.0,664.0){\usebox{\plotpoint}}
\put(243.0,665.0){\rule[-0.200pt]{0.723pt}{0.400pt}}
\put(246.0,665.0){\usebox{\plotpoint}}
\put(246.0,666.0){\rule[-0.200pt]{0.964pt}{0.400pt}}
\put(250.0,666.0){\usebox{\plotpoint}}
\put(252,666.67){\rule{0.241pt}{0.400pt}}
\multiput(252.00,666.17)(0.500,1.000){2}{\rule{0.120pt}{0.400pt}}
\put(250.0,667.0){\rule[-0.200pt]{0.482pt}{0.400pt}}
\put(253,668){\usebox{\plotpoint}}
\put(253,668){\usebox{\plotpoint}}
\put(253,668){\usebox{\plotpoint}}
\put(253,668){\usebox{\plotpoint}}
\put(253,668){\usebox{\plotpoint}}
\put(253,668){\usebox{\plotpoint}}
\put(253.0,668.0){\usebox{\plotpoint}}
\put(254.0,668.0){\usebox{\plotpoint}}
\put(254.0,669.0){\usebox{\plotpoint}}
\put(255.0,669.0){\usebox{\plotpoint}}
\put(255.0,670.0){\usebox{\plotpoint}}
\put(256.0,670.0){\usebox{\plotpoint}}
\put(256.0,671.0){\usebox{\plotpoint}}
\put(257.0,671.0){\usebox{\plotpoint}}
\put(257.0,672.0){\rule[-0.200pt]{0.723pt}{0.400pt}}
\put(260,672.67){\rule{0.241pt}{0.400pt}}
\multiput(260.00,672.17)(0.500,1.000){2}{\rule{0.120pt}{0.400pt}}
\put(260.0,672.0){\usebox{\plotpoint}}
\put(261,674){\usebox{\plotpoint}}
\put(261,674){\usebox{\plotpoint}}
\put(261,674){\usebox{\plotpoint}}
\put(261,674){\usebox{\plotpoint}}
\put(261,674){\usebox{\plotpoint}}
\put(261,674){\usebox{\plotpoint}}
\put(261.0,674.0){\rule[-0.200pt]{0.964pt}{0.400pt}}
\put(265.0,674.0){\usebox{\plotpoint}}
\put(265.0,675.0){\rule[-0.200pt]{0.482pt}{0.400pt}}
\put(267.0,675.0){\usebox{\plotpoint}}
\put(267.0,676.0){\usebox{\plotpoint}}
\put(268.0,676.0){\usebox{\plotpoint}}
\put(268.0,677.0){\usebox{\plotpoint}}
\put(269.0,677.0){\usebox{\plotpoint}}
\put(269.0,678.0){\rule[-0.200pt]{0.723pt}{0.400pt}}
\put(272.0,678.0){\usebox{\plotpoint}}
\put(272.0,679.0){\usebox{\plotpoint}}
\put(273.0,679.0){\usebox{\plotpoint}}
\put(273.0,680.0){\usebox{\plotpoint}}
\put(274.0,680.0){\usebox{\plotpoint}}
\put(274.0,681.0){\rule[-0.200pt]{0.723pt}{0.400pt}}
\put(277.0,681.0){\usebox{\plotpoint}}
\put(277.0,682.0){\rule[-0.200pt]{0.964pt}{0.400pt}}
\put(281.0,682.0){\rule[-0.200pt]{0.400pt}{0.482pt}}
\put(281.0,684.0){\usebox{\plotpoint}}
\put(282.0,684.0){\usebox{\plotpoint}}
\put(282.0,685.0){\rule[-0.200pt]{1.927pt}{0.400pt}}
\put(290.0,685.0){\usebox{\plotpoint}}
\put(290.0,686.0){\rule[-0.200pt]{0.723pt}{0.400pt}}
\put(293.0,686.0){\usebox{\plotpoint}}
\put(293.0,687.0){\rule[-0.200pt]{0.964pt}{0.400pt}}
\put(297.0,687.0){\usebox{\plotpoint}}
\put(297.0,688.0){\rule[-0.200pt]{4.336pt}{0.400pt}}
\put(315.0,688.0){\usebox{\plotpoint}}
\put(315.0,689.0){\rule[-0.200pt]{6.504pt}{0.400pt}}
\put(342.0,689.0){\usebox{\plotpoint}}
\put(355,689.67){\rule{0.241pt}{0.400pt}}
\multiput(355.00,689.17)(0.500,1.000){2}{\rule{0.120pt}{0.400pt}}
\put(342.0,690.0){\rule[-0.200pt]{3.132pt}{0.400pt}}
\put(356,691){\usebox{\plotpoint}}
\put(356,691){\usebox{\plotpoint}}
\put(356,691){\usebox{\plotpoint}}
\put(356,691){\usebox{\plotpoint}}
\put(356,691){\usebox{\plotpoint}}
\put(356,691){\usebox{\plotpoint}}
\put(356.0,691.0){\rule[-0.200pt]{188.143pt}{0.400pt}}
\put(1137.0,691.0){\rule[-0.200pt]{0.400pt}{5.059pt}}
\sbox{\plotpoint}{\rule[-0.500pt]{1.000pt}{1.000pt}}%
\sbox{\plotpoint}{\rule[-0.200pt]{0.400pt}{0.400pt}}%
\put(1058,171){\makebox(0,0)[r]{VSZ}}
\sbox{\plotpoint}{\rule[-0.500pt]{1.000pt}{1.000pt}}%
\multiput(1078,171)(20.756,0.000){5}{\usebox{\plotpoint}}
\put(1178,171){\usebox{\plotpoint}}
\put(171,224){\usebox{\plotpoint}}
\put(171.00,224.00){\usebox{\plotpoint}}
\multiput(172,243)(0.000,20.756){2}{\usebox{\plotpoint}}
\multiput(172,271)(0.000,20.756){15}{\usebox{\plotpoint}}
\multiput(172,590)(0.000,20.756){6}{\usebox{\plotpoint}}
\put(174.00,719.13){\usebox{\plotpoint}}
\put(177.47,737.00){\usebox{\plotpoint}}
\put(189.92,746.18){\usebox{\plotpoint}}
\put(199.00,757.93){\usebox{\plotpoint}}
\put(212.69,765.00){\usebox{\plotpoint}}
\put(233.45,765.00){\usebox{\plotpoint}}
\put(254.20,765.00){\usebox{\plotpoint}}
\put(274.96,765.00){\usebox{\plotpoint}}
\put(295.71,765.00){\usebox{\plotpoint}}
\put(316.47,765.00){\usebox{\plotpoint}}
\put(337.22,765.00){\usebox{\plotpoint}}
\put(357.98,765.00){\usebox{\plotpoint}}
\put(378.73,765.00){\usebox{\plotpoint}}
\put(399.49,765.00){\usebox{\plotpoint}}
\put(420.24,765.00){\usebox{\plotpoint}}
\put(441.00,765.00){\usebox{\plotpoint}}
\put(461.76,765.00){\usebox{\plotpoint}}
\put(482.51,765.00){\usebox{\plotpoint}}
\put(503.27,765.00){\usebox{\plotpoint}}
\put(524.02,765.00){\usebox{\plotpoint}}
\put(544.78,765.00){\usebox{\plotpoint}}
\put(565.53,765.00){\usebox{\plotpoint}}
\put(586.29,765.00){\usebox{\plotpoint}}
\put(607.04,765.00){\usebox{\plotpoint}}
\put(627.80,765.00){\usebox{\plotpoint}}
\put(648.56,765.00){\usebox{\plotpoint}}
\put(669.31,765.00){\usebox{\plotpoint}}
\put(690.07,765.00){\usebox{\plotpoint}}
\put(710.82,765.00){\usebox{\plotpoint}}
\put(731.58,765.00){\usebox{\plotpoint}}
\put(752.33,765.00){\usebox{\plotpoint}}
\put(773.09,765.00){\usebox{\plotpoint}}
\put(793.84,765.00){\usebox{\plotpoint}}
\put(814.60,765.00){\usebox{\plotpoint}}
\put(835.35,765.00){\usebox{\plotpoint}}
\put(856.11,765.00){\usebox{\plotpoint}}
\put(876.87,765.00){\usebox{\plotpoint}}
\put(897.62,765.00){\usebox{\plotpoint}}
\put(918.38,765.00){\usebox{\plotpoint}}
\put(939.13,765.00){\usebox{\plotpoint}}
\put(959.89,765.00){\usebox{\plotpoint}}
\put(980.64,765.00){\usebox{\plotpoint}}
\put(1001.40,765.00){\usebox{\plotpoint}}
\put(1022.15,765.00){\usebox{\plotpoint}}
\put(1042.91,765.00){\usebox{\plotpoint}}
\put(1063.67,765.00){\usebox{\plotpoint}}
\put(1084.42,765.00){\usebox{\plotpoint}}
\put(1105.18,765.00){\usebox{\plotpoint}}
\put(1125.93,765.00){\usebox{\plotpoint}}
\put(1137.00,755.31){\usebox{\plotpoint}}
\put(1137,754){\usebox{\plotpoint}}
\sbox{\plotpoint}{\rule[-0.200pt]{0.400pt}{0.400pt}}%
\put(171.0,131.0){\rule[-0.200pt]{0.400pt}{155.380pt}}
\put(171.0,131.0){\rule[-0.200pt]{252.222pt}{0.400pt}}
\put(1218.0,131.0){\rule[-0.200pt]{0.400pt}{155.380pt}}
\put(171.0,776.0){\rule[-0.200pt]{252.222pt}{0.400pt}}
\end{picture}

        \end{center}
    \end{figure}

\end{document}
